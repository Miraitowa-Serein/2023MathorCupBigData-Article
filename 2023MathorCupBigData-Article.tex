% !TeX program = xelatex
% !TeX encoding = UTF-8
\documentclass{MathorCupmodeling}
\usepackage{mwe,color,float}
\usepackage[linesnumbered,ruled]{algorithm2e}
\usepackage{setspace}
\usepackage{pdfpages}
\usepackage{graphicx}
\extrafloats{100}
\bianhao{MCB2301959}
\tihao{A}
\timu{\textbf{基于图形增强的CNN-PCA-SVM模型在图像分类中的应用}}
\keyword{特征工程;计算机视觉;支持向量机;评分预测;可视化评估}
\begin{document}
	\begin{abstract}
		{\heiti 最后},本文对所建立的模型的优缺点进行了中肯的评价、提出了模型的改进措施以及对模型进行了一定推广。
	\end{abstract}

	\pagestyle{empty}
	\tableofcontents
	\newpage
	\pagestyle{fancy}

	\setcounter{page}{1}
	\section{问题的提出}
	\subsection{问题背景}
	坑洼道路的检测与识别工作是推动自动无人驾驶、地质勘探、航天科学及自然灾害等领域研究和应用的不可或缺的计算机视觉任务。然而传统的分类算法往往因坑洼图像的复杂及多变性,不能取得较好的效果。近年来,深度学习邻域技术的发展为坑洼道路的检测、为当前存在的亟待解决的问题提供了新的思想。

	在实际工程应用中,非结构化的坑洼道路检测将遇到复杂多样的环境,例如,道路周围环境存在差异、光照条件变化等,这无疑增加了基于视觉检测道路的难度\textcolor{blue}{\cite{曹江华}}。道路坑洼的检测是实现智能车辆导航的基础,只有快速准确地检测出道路的可行驶区域才能够真正实现自动驾驶。

	深度学习拥有很强的特征提取与表示能力,可从图像中提取重要特征。于坑洼道路检测和识别而言,其可识别坑洼的轮廓、纹理、形态等特征,并将上述特征转换为更易分类的表现形式。在深度学习的基础上,迁移学习、知识蒸馏等技术可进一步提升分类性能以更加准确对新出现的道路图像自动识别。
	\subsection{问题要求}
	\begin{itemize}
		\item \textbf{问题一:}基于图像文件,提取图像特征,以“正常”和“坑洼”为特征建立一个识别率高、速度快且分类准确的模型。
		\item \textbf{问题二:}对问题一中所建立的模型进行训练,并对其进行多维度的评估分析。
		\item \textbf{问题三:}利用训练模型识别测试集中的坑洼图像,并展示结果。
	\end{itemize}
	\section{问题的分析}
	\subsection{问题的整体分析}
	该题是一个基于计算机视觉的坑洼道路的图像数据分析、预测类问题。

	\textbf{从分析目的看},

	\textbf{从数据来源、特征看},
	
	\textbf{从模型的选择看},

	\textbf{从编程软件的选择看},本题为图像大数据分析类,需要对大量的图片数据进行分析,并依据设问建立合适的模型,对正常及坑洼道路的图像进行分类预测,因此我们选择使用Python Jupyter对问题进行求解,其交互式的编程范式及轻量化,方便且高效。

	\textbf{从Python第三方库的选择看},

	\subsection{问题一的分析}
	问题一的核心目的在于\textbf{对原数据集进行重采样,并进行更深层次的分析}。
	\subsection{问题二的分析}
	问题二的核心目的在于\textbf{为移动公司撰写一份非技术性报告,为其提供合理性建议,从而为客户提供更好的服务}。
	\section{符号说明}
	\begin{center}
		\scalebox{0.92}{
		\begin{tabularx}{0.7\textwidth}{c@{\hspace{1pc}}|@{\hspace{2pc}}X}
			\Xhline{0.08em}
			符号 & \multicolumn{1}{c}{符号说明}\\
			\Xhline{0.05em}
			$\mu$ & 样本平均值\\
			$\sigma$ & 样本方差\\
			$x_{\mathrm{standard}}$ & 经过标准化后的数据\\
			$R\left(x\right)_{m\times n}$ & 经过某项处理后的数据特征集\\
			$\rho$ & 皮尔逊相关系数\\
			$x'$ & 经过某项处理后的数据\\
			$Gini$ & 样本集合基尼系数\\
			$\hat{y}$ & 预测值\\
			$L^{\left(t\right)}$ & 目标函数\\
			$\Omega$ & 叶节点正则项惩罚系数\\
			$P$ & 某事件发生的概率\\
			$\omega$ & 权重\\
			\Xhline{0.08em}
		\end{tabularx}}
	\end{center}
	\section{模型的假设}
	本文对于模型的假设与初赛假设一致,如下:
	\begin{itemize}
		\item \textbf{假设一}:语音与上网业务的八项评分中,存在个别用户乱评、错评现象;
		\item \textbf{假设二}:除个别用户的部分评分外,其余所有数据真实且符合实际情况;
		\item \textbf{假设三}:用户评分还受到除附件中因素之外的因素的影响;
		\item \textbf{假设四}:给定的数据集可全面体现用户整体情况;
		\item \textbf{假设五}:对于同一业务,学习数据与预测数据的内在规律是一致的。
	\end{itemize}
	\section{模型的建立与求解}
	
	\newpage
	
	\phantomsection
	\addcontentsline{toc}{section}{\textbf{参考文献}}
	\begin{spacing}{1.08}
	\begin{thebibliography}{99}
	\bibitem{曹江华}曹江华. 复杂背景下非结构化道路可行驶区域检测研究[D].浙江科技学院,2021.

	\bibitem{pstandard}CSDN.【数据预处理】sklearn实现数据预处理(归一化、标准化)[EB/OL].
	
	\url{https://blog.csdn.net/weixin_44109827/article/details/124786873}.


	\bibitem{ppearson}王殿武,赵云斌,尚丽英,王凤刚,张震.皮尔逊相关系数算法在B油田优选化学防砂措施井的应用[J].精细与专用化学品,2022,30(07):26-28.DOI:10.19482/j.cn11-3237.2022.07.07.

	\bibitem{prf}饶雷,冉军,陶建权,胡号朋,吴沁,熊圣新.基于随机森林的海上风电机组发电机轴承异常状态监测方法[J].船舶工程,2022,44(S2):27-31.DOI:10.13788/j.cnki.cbgc.2022.S2.06.

	\bibitem{pxgboost1}陈振宇,刘金波,李晨,季晓慧,李大鹏,黄运豪,狄方春,高兴宇,徐立中.基于LSTM与XGBoost组合模型的超短期电力负荷预测[J].电网技术,2020,44(02):614-620.DOI:10.13335/j.1000-3673.pst.2019.1566.

	\bibitem{pxgboost2}杨贵军,徐雪,赵富强.基于XGBoost算法的用户评分预测模型及应用[J].数据分析与知识发现,2019,3(01):118-126.

	\bibitem{pxgboost3}Tianqi Chen and Carlos Guestrin. 2016. XGBoost: A Scalable Tree Boosting System. In Proceedings of the 22nd ACM SIGKDD International Conference on Knowledge Discovery and Data Mining (KDD '16). Association for Computing Machinery, New York, NY, USA, 785–794. \url{https://doi.org/10.1145/2939672.2939785}.

	\bibitem{pknn}张著英,黄玉龙,王翰虎.一个高效的KNN分类算法[J].计算机科学,2008(03):170-172.

	\bibitem{psvm}汪海燕,黎建辉,杨风雷.支持向量机理论及算法研究综述[J].计算机应用研究,2014,31(05):1281-1286.

	\bibitem{plightgbm}马晓君,沙靖岚,牛雪琪.基于LightGBM算法的P2P项目信用评级模型的设计及应用[J].数量经济技术经济研究,2018,35(05):144-160.DOI:10.13653/j.cnki.jqte.20180503.001.

	\bibitem{plr}唐敏,张宇浩,邓国强.高效的非交互式隐私保护逻辑回归模型[J/OL].计算机工程:1-11[2023-01-04].DOI:10.19678/j.issn.1000-3428.0065549.

	\bibitem{pstacking}史佳琪,张建华.基于多模型融合Stacking集成学习方式的负荷预测方法[J].中国电机工程学报,2019,39(14):4032-4042.DOI:10.13334/j.0258-8013.pcsee.181510.

	\bibitem{procauc}A.Tharwat, Applied Computing and Informatics (2018). \url{https://doi.org/10.1016/j.aci.2018.08.003}.
	\end{thebibliography}
	\end{spacing}
	\newpage

	\phantomsection
	\addcontentsline{toc}{section}{\textbf{附\hspace{2pc}录}}

	% \appendix
	% \ctexset{section={format={\zihao{-4}\heiti\raggedright}}}
	\begin{center}
		\heiti\zihao{4} 附\hspace{2pc}录
	\end{center}

% \phantomsection
% \addcontentsline{toc}{subsection}{[A]图示}
% 	% \section*{[A]图表}
	\noindent{\heiti [A]图示}
\newpage
% \phantomsection
% \addcontentsline{toc}{subsection}{[B]支撑文件列表}
% 	% \section*{[B]支撑文件列表}
	\noindent{\heiti [B]支撑文件列表}
	~\\

	支撑文件列表如下(列表中不包含原始数据集):
	% Table generated by Excel2LaTeX from sheet 'Sheet2'
	\begin{table}[htbp]
	\centering
	\scalebox{0.88}{
	  \begin{tabular}{cc}
	  \toprule
	  \textbf{文件(夹)名} & \textbf{描述} \\
	  \midrule
	  附件6:result.xlsx & 用户评分预测结果 \\
	  语音业务高分组描述.csv & 语音业务高分组数据描述 \\
	  语音业务低分组描述.csv & 语音业务低分组数据描述 \\
	  上网业务高分组描述.csv & 上网业务高分组数据描述 \\
	  上网业务低分组描述.csv & 上网业务低分组数据描述 \\
	  语音业务Sample.csv & 语音业务剔除不合理评分后样本数据 \\
	  上网业务Sample.csv & 上网业务剔除不合理评分后样本数据 \\
	  语音业务词云.txt & 语音业务词云图文本内容 \\
	  上网业务词云.txt & 上网业务词云图文本内容 \\
	  词云.txt & 语音及上网业务综合词云图文本内容 \\
	  词云图.py & 语音及上网业务综合词云图代码文件 \\
	  语音业务用户分析.ipynb & 语音业务评分特征分析Jupyter文件 \\
	  上网业务用户分析.ipynb & 上网业务评分特征分析Jupyter文件 \\
	  语音业务数据分析.ipynb & 语音业务模型建立Jupyter文件 \\
	  上网业务数据分析.ipynb & 上网业务模型建立Jupyter文件 \\
	  语音业务用户分析.html & 语音业务评分特征分析运行结果 \\
	  上网业务用户分析.html & 上网业务评分特征分析运行结果 \\
	  语音业务数据分析.html & 语音业务模型建立运行结果 \\
	  上网业务数据分析.html & 上网业务模型建立运行结果 \\
	  bg.jpg & 词云底图 \\
	  wordcloud.png & 语音及上网业务综合词云图 \\
	  figuresOne & 语音业务所有图示文件夹 \\
	  figuresTwo & 上网业务所有图示文件夹 \\
	  \bottomrule
	  \end{tabular}}
  	\end{table}
  
\newpage
% \phantomsection
% \addcontentsline{toc}{subsection}{[C]使用的软件、环境}
	% \section*{[C]使用的软件、环境}
	\noindent{\heiti [C]使用的软件、环境}
	~\\

	为解决该问题,我们所使用的主要软件有:
	\begin{itemize}
		\item TeX Live 2022
		\item Visual Studio Code 1.83.1
		\item WPS Office 2023秋季更新(15398)
		\item Python 3.10.4 [MSC v.1929 64 bit (AMD64)] on win32
		\item Pycharm 2023.2.3 (Professional Edition)
	\end{itemize}
	
	Python环境下所用使用到的库及其版本如下:
\begin{table}[htbp]
	\centering
	\setlength{\aboverulesep}{0pt}
	\setlength{\belowrulesep}{0pt}
	\scalebox{0.88}{
	  \begin{tabular}{cc||cc}
	  \toprule
	  \textbf{库} & \textbf{版本} & \textbf{库} & \textbf{版本} \\
	  \midrule
	  copy  & 内置库   & missingno & 0.5.1 \\
	  jieba & 0.42.1 & mlxtend & 0.20.2 \\
	  jupyter & 1.0.0 & numpy & 1.22.4+mkl \\
	  jupyter-client & 7.3.1 & openpyxl & 3.0.10 \\
	  jupyter-console & 6.4.3 & pandas & 1.4.2 \\
	  jupyter-contrib-core & 0.4.0 & pyecharts & 1.9.1 \\
	  jupyter-contrib-nbextensions & 0.5.1 & scikit-learn & 0.22.2.post1 \\
	  jupyter-core & 4.10.0 & seaborn & 0.11.2 \\
	  jupyter-highlight-selected-word & 0.2.0 & sklearn & 0.0  \\
	  jupyterlab-pygments & 0.2.2 & snapshot\_phantomjs & 0.0.3 \\
	  jupyterlab-widgets & 1.1.0 & warnings & 内置库 \\
	  jupyter-latex-envs & 1.4.6 & wordcloud & 1.8.1 \\
	  jupyter-nbextensions-configurator & 0.5.0 & xgboost & 1.6.1 \\
	  matplotlib & 3.5.2 & yellowbrick & 1.4 \\
	  \bottomrule
	  \end{tabular}}
\end{table}

\newpage
% \phantomsection
% \addcontentsline{toc}{subsection}{[D]问题解决源程序}
	% \section*{[D]问题解决源程序}
\noindent{\heiti [D]问题解决源程序}

\textbf{D.1 Data Preprocessing 数据预处理}
\begin{python}
#!/usr/bin/env python
# coding: utf-8

# In[1]:


import os
import shutil

# 指定目录"DATA"
path = "DATA"

# 在"DATA"文件夹中创建"normal"和"potholes"文件夹
os.mkdir(os.path.join(path, "normal"))
os.mkdir(os.path.join(path, "potholes"))

# 读取"DATA"文件夹,若文件名中含有"normal",则将其放置于"normal"文件夹中,否则放置于"potholes"文件夹中
files = os.listdir(path)
for file in files:
    if "normal" in file:
        shutil.move(os.path.join(path, file), os.path.join(path, "normal"))
    else:
        shutil.move(os.path.join(path, file), os.path.join(path, "potholes"))

\end{python}

\textbf{D.2 Comparative Analysis Of Normal And Potholes 正常与坑洼道路的比较分析}
\begin{python}
#!/usr/bin/env python
# coding: utf-8

# In[1]:


import cv2
import numpy as np
import matplotlib.pyplot as plt

# In[2]:


normalImg = cv2.imread('DATA\\normal\\normal133.jpg')
potholesImg = cv2.imread('DATA\\potholes\\potholes1.jpg')

# In[3]:


plt.rcParams['font.sans-serif'] = ['Times New Roman']
plt.rcParams['axes.unicode_minus'] = False


# In[4]:


def cv_show(img):
    b, g, r = cv2.split(img)
    img = cv2.merge([r, g, b])
    plt.imshow(img)


# In[5]:


cv_show(normalImg)

# In[6]:


cv_show(potholesImg)

# In[7]:


color = ('b', 'g', 'r')

for i, col in enumerate(color):
    histr = cv2.calcHist([normalImg], [i], None, [256], [0, 256])
    plt.plot(histr, color=col)

plt.legend(['Blue', 'Green', 'Red'])
plt.xlim([0, 256])
plt.xticks(fontsize=10)
plt.yticks(fontsize=10)
plt.title('(a) normal133.jpg', y=-0.2, fontsize=12)
plt.xlabel('Pixel Value', fontsize=11)
plt.ylabel('Number of Pixels', fontsize=11)
plt.savefig('Figures\\normal133RGB直方图.pdf', bbox_inches='tight')

# In[8]:


color = ('b', 'g', 'r')

for i, col in enumerate(color):
    histr = cv2.calcHist([potholesImg], [i], None, [256], [0, 256])
    plt.plot(histr, color=col)

plt.legend(['Blue', 'Green', 'Red'])
plt.xlim([0, 256])
plt.xticks(fontsize=10)
plt.yticks(fontsize=10)
plt.title('(b) potholes1.jpg', y=-0.2, fontsize=12)
plt.xlabel('Pixel Value', fontsize=11)
plt.ylabel('Number of Pixels', fontsize=11)
plt.savefig('Figures\\potholes1RGB直方图.pdf', bbox_inches='tight')

# In[9]:


plt.style.use('ggplot')
plt.hist(normalImg.ravel(), 256, [0, 256], color='grey')
plt.title('(a) normal133.jpg', y=-0.2, fontsize=12)
plt.xlabel('Pixel Value', fontsize=11)
plt.ylabel('Number of Pixels', fontsize=11)
plt.savefig('Figures\\normal133灰度直方图.pdf', bbox_inches='tight')

# In[10]:


plt.style.use('ggplot')
plt.hist(potholesImg.ravel(), 256, [0, 256], color='grey')
plt.title('(b) potholes1.jpg', y=-0.2, fontsize=12)
plt.xlabel('Pixel Value', fontsize=11)
plt.ylabel('Number of Pixels', fontsize=11)
plt.savefig('Figures\\potholes1灰度直方图.pdf', bbox_inches='tight')

# In[11]:


# 边缘检测
gray = cv2.cvtColor(normalImg, cv2.COLOR_BGR2GRAY)
edges = cv2.Canny(gray, 100, 200)
plt.imshow(edges, cmap='gray')
plt.title('(a) normal133.jpg', y=-0.2, fontsize=12)
plt.savefig('Figures\\normal133边缘检测.pdf', bbox_inches='tight')

# In[12]:


# 边缘检测
gray = cv2.cvtColor(potholesImg, cv2.COLOR_BGR2GRAY)
edges = cv2.Canny(gray, 100, 200)
plt.imshow(edges, cmap='gray')
plt.title('(b) potholes1.jpg', y=-0.2, fontsize=12)
plt.savefig('Figures\\potholes1边缘检测.pdf', bbox_inches='tight')

# In[13]:


plt.imshow(normalImg)
plt.colorbar()
plt.title('(a) normal133.jpg', y=-0.3, fontsize=12)
plt.savefig('Figures\\normal133热力图.pdf')

# In[14]:


plt.imshow(potholesImg)
plt.colorbar()
plt.title('(b) potholes1.jpg', y=-0.3, fontsize=12)
plt.savefig('Figures\\potholes1热力图.pdf')

# In[15]:


# 阈值分割
hsv = cv2.cvtColor(normalImg, cv2.COLOR_BGR2HSV)
lower_blue = np.array([90, 50, 50])
upper_blue = np.array([130, 255, 255])
mask = cv2.inRange(hsv, lower_blue, upper_blue)
plt.imshow(mask, cmap='gray')
plt.title('(a) normal133.jpg', y=-0.2, fontsize=12)
plt.savefig('Figures\\normal133阈值分割.pdf', bbox_inches='tight')

# In[16]:


# 阈值分割
hsv = cv2.cvtColor(potholesImg, cv2.COLOR_BGR2HSV)
lower_blue = np.array([90, 50, 50])
upper_blue = np.array([130, 255, 255])
mask = cv2.inRange(hsv, lower_blue, upper_blue)
plt.imshow(mask, cmap='gray')
plt.title('(b) potholes1.jpg', y=-0.2, fontsize=12)
plt.savefig('Figures\\potholes1阈值分割.pdf', bbox_inches='tight')

# In[17]:


# 转换为灰度图像
gray = cv2.cvtColor(normalImg, cv2.COLOR_BGR2GRAY)
# 二值化
ret, binary = cv2.threshold(gray, 0, 255, cv2.THRESH_BINARY | cv2.THRESH_OTSU)
# 轮廓检测
contours, hierarchy = cv2.findContours(binary, cv2.RETR_TREE, cv2.CHAIN_APPROX_SIMPLE)
# 绘制轮廓
cv2.drawContours(normalImg, contours, -1, (0, 0, 255), 3)

plt.subplot(1, 2, 1)
plt.title('(a) Original', y=-0.4, fontsize=12)
plt.imshow(normalImg)
plt.subplot(1, 2, 2)
plt.imshow(binary, cmap='gray')
plt.title('(b) Contours', y=-0.4, fontsize=12)
plt.savefig('Figures\\normal133轮廓检测.pdf', bbox_inches='tight')

# In[18]:


# 转换为灰度图像
gray = cv2.cvtColor(potholesImg, cv2.COLOR_BGR2GRAY)
# 二值化
ret, binary = cv2.threshold(gray, 0, 255, cv2.THRESH_BINARY | cv2.THRESH_OTSU)
# 轮廓检测
contours, hierarchy = cv2.findContours(binary, cv2.RETR_TREE, cv2.CHAIN_APPROX_SIMPLE)
# 绘制轮廓
cv2.drawContours(potholesImg, contours, -1, (0, 0, 255), 3)

plt.subplot(1, 2, 1)
plt.title('(a) Original', y=-0.4, fontsize=12)
plt.imshow(potholesImg)
plt.subplot(1, 2, 2)
plt.imshow(binary, cmap='gray')
plt.title('(b) Contours', y=-0.4, fontsize=12)
plt.savefig('Figures\\potholes1轮廓检测.pdf', bbox_inches='tight')

\end{python}

\textbf{D.3 Random Plot Images 随机展现正常与坑洼道路图像}
\begin{python}
#!/usr/bin/env python
# coding: utf-8

# In[1]:


import pathlib
import matplotlib.pyplot as plt
from keras.preprocessing.image import ImageDataGenerator


# In[2]:


dataDirectory = pathlib.Path('DATA\\')
classNames = [item.name for item in dataDirectory.glob('*')][:2]
classNames


# In[3]:


dataAdd = 'DATA' 
normalAdd = 'DATA\\normal'
potholesAdd = 'DATA\\potholes'

# 定义一个数据生成器,用于处理图像数据,并对数据进行归一化处理。同时,将数据集的20%作为验证数据,而其余80%用于训练
dataImageDataGenerator = ImageDataGenerator(rescale = 1/255., validation_split = 0.2)
dataTrain = dataImageDataGenerator.flow_from_directory(dataAdd, target_size = (224, 224), batch_size = 32, subset = 'training', class_mode = 'binary')
dataVal = dataImageDataGenerator.flow_from_directory(dataAdd, target_size = (224, 224), batch_size = 32, subset = 'validation', class_mode = 'binary')


# In[4]:


def random_plot_images():
    images, labels = dataTrain.next()
    labels = labels.astype('int32')
    i = 1

    plt.figure(figsize = (10, 10))
    
    for image, label in zip(images, labels):
        plt.subplot(4, 5, i)
        plt.imshow(image)
        if label == 0:
            plt.title(classNames[label],fontname='Times New Roman', fontsize=12, color='blue')
        else:
            plt.title(classNames[label],fontname='Times New Roman', fontsize=12, color='red')
        plt.axis('off')
        i += 1
        if i == 21:
            break

    plt.tight_layout()       
    plt.savefig('Figures\\图像数据观测.pdf')


random_plot_images()

\end{python}

\textbf{D.4 PCA-SVM PCA降维、网格调优的支持向量机}
\begin{python}
#!/usr/bin/env python
# coding: utf-8

# In[1]:


import os
import cv2
import numpy as np
import collections
from sklearn import svm
import matplotlib.pyplot as plt
from sklearn.decomposition import PCA
from sklearn.metrics import accuracy_score
from keras.preprocessing.image import ImageDataGenerator


# In[2]:


# 加载图像数据
def load_img():
    inputImg, inputLabel = [], []
    resize = (224, 224)
    for dirname, _, filenames in os.walk('DATA\\'):
        for filename in filenames:
            photo_path = os.path.join(dirname, filename)
            photo_class = dirname.split('\\')[-1]
            try:
                read_im = cv2.imread(photo_path)
                inputImg.append(cv2.resize(read_im, resize))
                # potholes == 0
                if photo_class == 'potholes':
                    inputLabel.append(0)
                # normal == 1
                elif photo_class == 'normal':
                    inputLabel.append(1)
            except:
                print(photo_path)
    return inputImg, inputLabel


inputImg, inputLabel = load_img()

# In[3]:


# 计算inputLabel中各类别的数量
counter = collections.Counter(inputLabel)
counter


# In[4]:


# 随机划分训练集和测试集,比例为test_prop,x为图像数据,y为标签
def train_test_split(test_prop, inputImg, inputLabel):
    test_size = int(np.floor(test_prop * len(inputLabel)))
    # 随机数
    np.random.seed(202310)
    test_index = np.random.choice(len(inputLabel), size=test_size, replace=False)
    # 划分
    train_x, test_x, train_y, test_y = np.delete(inputImg, test_index, axis=0), np.take(inputImg, test_index, axis=0), np.delete(inputLabel, test_index, axis=0), np.take(inputLabel, test_index, axis=0)
    # 返回图像和标签的训练集和测试集
    return train_x, test_x, train_y, test_y, test_index


train_x, test_x, train_y, test_y, test_index = train_test_split(0.2, inputImg, inputLabel)


# In[5]:


# opencv滤波增加训练集
def opencv_blur(inputImg, inputLabel):
    inputLabelNew = inputLabel.copy()
    inputImgNew = inputImg.copy()
    for i in range(len(inputImg)):
        im = inputImg[i]
        im = im.astype('uint8')
        imLbl = [inputLabel[i]]

        # 高斯滤波
        imgGaussian = cv2.GaussianBlur(im, (5, 5), 0)
        # 双边滤波
        imgBilateral = cv2.bilateralFilter(im, 9, 75, 75)

        # 添加到训练集中
        inputImgNew = np.append(inputImgNew, [imgGaussian, imgBilateral], axis=0)
        inputLabelNew = np.append(inputLabelNew, imLbl * 2, axis=0)
    return inputImgNew, inputLabelNew


inputImgNew, inputLabelNew = opencv_blur(train_x, train_y)


# In[6]:


# 图像增强,在原有的训练集上进行图像增强
def append_img(inputImg, inputLabel, imgIterator):
    inputLabelNew = inputLabel.copy()
    inputImgNew = inputImg.copy()
    for i in range(len(imgIterator)):
        im = imgIterator[i]
        im = im.astype('uint8')
        imLbl = [inputLabel[i]]
        inputImgNew = np.append(inputImgNew, im, axis=0)
        inputLabelNew = np.append(inputLabelNew, imLbl, axis=0)
    return inputImgNew, inputLabelNew


# In[7]:


# 旋转 + 30 deg
rotate_data_generartor = ImageDataGenerator(rotation_range=30)
imgIterator = rotate_data_generartor.flow(train_x, batch_size=1, shuffle=False)
inputImgNew, inputLabelNew = append_img(inputImgNew, inputLabelNew, imgIterator)

# In[8]:


# 计算inputLabelNew中各类别的数量
counter = collections.Counter(inputLabelNew)
counter


# In[9]:


def plot_img(inputImgNew, inputLabelNew):
    plt.figure(figsize=(12, 5))
    i = 1
    for image, label in zip(inputImgNew, inputLabelNew):
        if i <= 5:
            if label == 1:
                plt.subplot(2, 5, i)
                plt.imshow(image)
                plt.title('normal', fontname='Times New Roman', fontsize=12, color='blue')
                plt.axis('off')
                i += 1
        elif 5 < i < 11:
            if label == 0:
                plt.subplot(2, 5, i)
                plt.imshow(image)
                plt.title('potholes', fontname='Times New Roman', fontsize=12, color='red')
                plt.axis('off')
                i += 1
        else:
            break
    plt.savefig('Figures\\PCA-SVM训练样本.pdf', bbox_inches='tight')


plot_img(inputImgNew, inputLabelNew)

# In[10]:


nx, ny, nz = train_x.shape[1], train_x.shape[2], train_x.shape[3]
train_x_nn, test_x_nn = inputImgNew, test_x
train_x = inputImgNew.reshape((inputImgNew.shape[0], nx * ny * nz)) / 255
test_x = test_x.reshape((test_x.shape[0], nx * ny * nz)) / 255
train_y = inputLabelNew.reshape((inputLabelNew.shape[0], 1))
test_y = test_y.reshape((test_y.shape[0], 1))

# In[11]:


im_pca = PCA()
im_pca.fit(train_x)
variance_explained_list = im_pca.explained_variance_ratio_.cumsum()

# In[12]:


test_x_pca = im_pca.transform(test_x)
train_x_pca = im_pca.transform(train_x)


# In[13]:


# SVM网格调优
def svm_grid_search(C, kernel, train_x, train_y):
    accuracy_score_list = []

    for c in C:
        svmClassifier = svm.SVC(C=c, kernel=kernel)
        svmClassifier.fit(train_x, train_y.ravel())
        pred_y = svmClassifier.predict(train_x)
        accuracy = accuracy_score(train_y, pred_y)
        accuracy_score_list.append(accuracy)
        print('Regularization parameters: {:.2f},Accuracy:{:.4f}'.format(c, accuracy))

    max_accurarcy_id = accuracy_score_list.index(max(accuracy_score_list))
    return C[max_accurarcy_id]


C, kernel = [0.1 * i for i in range(1, 30)], 'rbf'
opt_C = svm_grid_search(C, kernel, train_x_pca, train_y)

# In[14]:


svmClassifier = svm.SVC(C=opt_C, kernel='rbf')
svmClassifier.fit(train_x_pca, train_y.ravel())
pred_y = svmClassifier.predict(test_x_pca)
accuracy = accuracy_score(test_y, pred_y)
print('Test Accuracy: {:2.2f}%'.format(accuracy * 100))

# In[15]:


# 分类报告
from sklearn.metrics import classification_report

print(classification_report(test_y, pred_y))

# In[16]:


# 保存模型
import pickle

with open('Models\\PCA.pkl', 'wb') as f:
    pickle.dump(im_pca, f)
with open('Models\\PCA-SVM.pkl', 'wb') as f:
    pickle.dump(svmClassifier, f)
f.close()

\end{python}

\textbf{D.5 CNN 卷积神经网络}
\begin{python}
#!/usr/bin/env python
# coding: utf-8

# In[1]:


import os
import cv2
import h5py
import warnings
import numpy as np
import collections
import random as rn
import matplotlib.pyplot as plt
from keras.models import Sequential
from keras.utils import to_categorical
from keras.layers import Conv2D, MaxPooling2D
from sklearn.preprocessing import LabelEncoder
from keras.layers import Dense, Flatten, Dropout
from sklearn.model_selection import train_test_split

warnings.filterwarnings('always')
warnings.filterwarnings('ignore')

# In[2]:


imagePath = []
for dirname, _, filenames in os.walk('DATAX\\'):
    # 统计DATAX文件夹下子文件夹下的图片数量
    print(dirname, len(filenames))
    for filename in filenames:
        path = os.path.join(dirname, filename)
        imagePath.append(path)

len(imagePath)

# In[3]:


IMG_SIZE = 128
X = []
y = []
for image in imagePath:
    try:
        img = cv2.imread(image, cv2.IMREAD_COLOR)
        img = cv2.resize(img, (IMG_SIZE, IMG_SIZE))
        X.append(np.array(img))
        if image.startswith('DATAX\\normal'):
            y.append('normal')
        else:
            y.append('potholes')
    except:
        pass

# In[4]:


fig, ax = plt.subplots(2, 5)
plt.subplots_adjust(bottom=0.3, top=0.7, hspace=0)
fig.set_size_inches(15, 12.5)

plt.rcParams['font.sans-serif'] = ['Times New Roman']
plt.rcParams['axes.unicode_minus'] = False

for i in range(2):
    for j in range(5):
        l = rn.randint(0, len(y))
        ax[i, j].imshow(X[l][:, :, ::-1])
        if y[l] == 'normal':
            ax[i, j].set_title(y[l], color='blue')
        else:
            ax[i, j].set_title(y[l], color='red')
        ax[i, j].set_aspect('equal')

plt.savefig('Figures\\CNN训练样本.pdf', bbox_inches='tight')

# In[5]:


# 统计y中各类别的数量
collections.Counter(y)

# In[6]:


le = LabelEncoder()
Y = le.fit_transform(y)
Y = to_categorical(Y, 2)
X = np.array(X)

# In[7]:


y

# In[8]:


# Y中第一列:1表示normal,0表示potholes
Y

# In[9]:


# 划分训练集和测试集,测试集占比20%
x_train, x_test, y_train, y_test = train_test_split(X, Y, test_size=0.20, random_state=5)

# In[10]:


model = Sequential()
model.add(Conv2D(32, (5, 5), activation='relu', input_shape=(128, 128, 3)))
model.add(MaxPooling2D((2, 2)))
model.add(Conv2D(64, (3, 3), activation='relu'))
model.add(MaxPooling2D((2, 2)))
model.add(Conv2D(128, (3, 3), activation='relu'))
model.add(MaxPooling2D((2, 2)))
model.add(Conv2D(128, (3, 3), activation='relu'))
model.add(MaxPooling2D((2, 2)))
model.add(Conv2D(128, (3, 3), activation='relu'))
model.add(MaxPooling2D((2, 2)))

model.add(Flatten())
model.add(Dropout(0.4))
model.add(Dense(128, activation='relu'))
model.add(Dense(2, activation='softmax'))

# In[11]:


model.compile(optimizer='adam', loss='categorical_crossentropy', metrics=['accuracy'])
model.summary()

# In[12]:


history = model.fit(x_train, y_train, epochs=50, batch_size=12, verbose=2, validation_data=(x_test, y_test))

# In[13]:


loss, accuracy = model.evaluate(x_test, y_test)
print('Test Accuracy: {:2.2f}%'.format(accuracy * 100))

# In[14]:


plt.figure(figsize=(12, 5))
plt.subplot(1, 2, 1)
plt.title('(a) Accuracy and Loss', y=-0.2)
plt.plot(history.history['accuracy'])
plt.plot(history.history['loss'])
plt.xlabel('Epochs')
plt.ylabel('Accuracy')
plt.legend(['Accuracy', 'Loss'], loc='upper right')
plt.subplot(1, 2, 2)
plt.title('(b) Validation Accuracy and Loss', y=-0.2)
plt.plot(history.history['val_accuracy'])
plt.plot(history.history['val_loss'])
plt.xlabel('Epochs')
plt.ylabel('Validation Accuracy and Loss')
plt.legend(['Val Accuracy', 'Val Loss'], loc='upper left')
plt.savefig('Figures\\CNN训练结果.pdf', bbox_inches='tight')

# In[15]:


prediction = model.predict(x_test)
y_pred = np.argmax(prediction, axis=1)

# In[16]:


y_testA = y_test.astype(int)
y_testB = []
for i in y_testA:
    a = 1
    if i[0] == 1 and i[1] == 0:
        a = 0
    y_testB.append(a)

# In[17]:


# 分类报告
from sklearn.metrics import classification_report

print(classification_report(y_testB, y_pred))

# In[18]:


# 保存数据集为h5文件,包括X_train, X_test, y_train, y_test
h5f = h5py.File('Models\\CNNData.h5', 'w')
h5f.create_dataset('X_train', data=x_train)
h5f.create_dataset('X_test', data=x_test)
h5f.create_dataset('y_train', data=y_train)
h5f.create_dataset('y_test', data=y_test)
h5f.close()

# In[19]:


# 保存模型
model.save('Models\\CNN.h5')

\end{python}
\end{document}